\documentclass[a4paper]{article}

%% Language and font encodings
\usepackage[russian]{babel}
\usepackage[utf8x]{inputenc}
\usepackage[T1]{fontenc}

%% Sets page size and margins
\usepackage[a4paper,top=3cm,bottom=2cm,left=3cm,right=3cm,marginparwidth=1.75cm]{geometry}

%% Useful packages
\usepackage{amsmath}
\usepackage{amssymb}
\usepackage{graphicx}
\usepackage[colorinlistoftodos]{todonotes}
\usepackage[colorlinks=true, allcolors=blue]{hyperref}

\title{Методы Оптимизации, ДЗ №5}
\author{Начинкин Илья, 695}

\begin{document}
\maketitle

\section*{Задача №1}
$$(x_1 - 3)^2 + (x_2 - 2)^2 \xrightarrow[]{} min$$
$$s.t. x_1 + 2x_2 = 4$$
    $$x_1^2 + x_2^2 \leq 5$$
Построить двойственную задачу (ДЗ), решить ее и по решению ДЗ найти решение прямой задачи.   
\subsection*{Решение:}

$L(x, \mu, \lambda) = (x_1 - 3)^2 + (x_2 - 2)^2 + \lambda(x_1 + 2x_2 - 4) + \mu(x_1^2 + x_2^2 - 5) \\
g(\mu,\lambda) = \inf_{x}L(x, \mu, \lambda)$ - двойств. функция. 
Посчитаем ее:

$\frac{\partial L}{\partial x_1} = 2x_1 - 6 + \lambda + 2\mu x_1 = 0 \Longrightarrow x_1 = \frac{6 - \lambda}{2(1+\mu)}$

$\frac{\partial L}{\partial x_2} = 2x_2 - 4 + 2\lambda + 2\mu x_2 = 0 \Longrightarrow x_2 = \frac{2 - \lambda}{1+\mu}$

Подставим эти значения в  лангранжиан и получим двойств. функцию

То есть двойств. функция $g(\mu, \lambda) = -\frac{5\lambda^2 + 16\lambda\mu - 12\lambda + 20\mu^2 - 32\mu}{4(\mu + 1)}$

Двойств. задача:
$$g(\mu, \lambda) \xrightarrow[]{} max$$
$$s.t. \mu \geq 0$$

Условие Слейтера выполнено , т.к. при $x_1 = 0, x_2 = 2$ выполнено $ x_1^2 + x_2^2 < 5$ - То есть строгое неравенство $\Longrightarrow$ сильная двойственность.

Ищем решение двойств. задачи:

$$\frac{\partial g}{\partial \lambda} = \frac{-5\lambda - 8\mu + 6}{2\mu + 2} = 0$$
$$\frac{\partial g}{\partial \mu} = \frac{5\lambda^2 - 28\lambda - 20\mu^2 - 40\mu + 32}{4(\mu + 1)^2} = 0$$

Решаем эту систему и получаем:
$\mu = -\frac{7}{3}$ - Но это неподходит под ограничение. А другое решение $ \\
\mu^* = \frac{1}{3}, \lambda^* = \frac{2}{3}$ - подходит. Это и есть решение двойств. задачи.
Тогда $x_1^* = 2, x_1^* = 1 \Longrightarrow f(x) = 2$ 

**Ответ:**$f(x) = 2$

\section*{Задача №2}
$$max(3x_1 + 2x_2)$$
$$s.t. x_1 + 3x_2 \leq 3$$
$$6x_1 - x_2 = 4$$
$$x_! + 2x_2 \leq 2$$
$$x_1 \geq 0; x_2 \geq 0$$
Найти зазор двойственности, решения прямой и двойств. задачи.
\subsection*{Решение:}
эта задача экв. нахождению $-min(-3x_1-2x_2)$

$L(x, \mu,\lambda) = -3x_1 -2x_2 + \lambda(6x_1 - x_2 - 4) + \mu_1(x_1 + 3x_2 -3) + \mu_2(x_1 + x_2 - 2) - \mu_3 x_1 - \mu_4 x_2 = x_1(-3 + 6\lambda + \mu_1 + \mu_2 - \mu_3) + x_2(-2 - \lambda + 3\mu_1 + 2\mu_2 -\mu_4) + (-4\lambda - 3\mu_1 - 2\mu_2)$

По теореме ККТ:

\begin{equation*}
    \begin{cases}
    $L'_{x_1} = -3 + 6\lambda + \mu_1 + \mu_2 - \mu_3 = 0$ \\
    $L'_{x_2} = -2 - \lambda + 3\mu_1 + 2\mu_2 -\mu_4 = 0$ \\
    $\mu_1(x_1 + 3x_2 - 3) = 0$ \\
    $\mu_2(x_1 + 2x_2 - 2) = 0$ \\
    $\mu_3 x_1 = 0$ \\
    $\mu_4 x_2 = 0$
    \end{cases}
\end{equation*}

Рассм. случай:
$\mu_4 > 0; x_2 = 0; \mu_1=\mu_2=\mu_3=0:$ \\

\begin{equation*}
    \begin{cases}
    $ -3 + 6\lambda = 0$ \\
    $-2 - \lambda + \mu_4 = 0$ \\
    $6x_1 = 4$
    \end{cases}
\end{equation*}

$x_1 = 2/3; x_2 = 0$ - минимум $ \Longleftarrow p^* = f_{min} = -2$

Теперь найдем решние двойств. задачи:

$g(\mu, \lambda) = \inf_{x}L(x_1, x_2, \lambda, \mu)$\\
Поскольку $x_i \geq 0 $, то $\inf_{x}$ отличен от $-\infty$, когда коэф-фы перед $x_i$ неотрицательны. А именно:

\begin{equation*}
    \begin{cases}
    $-3 + 6\lambda + \mu_1 \geq 0$ \\
    $-2 - \lambda + 3\mu_1 \geq 0$
    \end{cases}
\end{equation*}

То есть : $\mu_1 \geq \frac{15}{19}; \lambda \geq \frac{7}{19}$

Получаем:

\begin{equation*}
    g(\lambda, \mu_1, \mu_2) = \begin{cases}
    $-4\lambda - 3\mu_1 - 2\mu_2, \mu_1 \geq \frac{15}{19}, \lambda \geq \frac{7}{19}$\\
    $-\infty$, иначе
    \end{cases}
\end{equation*}

$\max_{\lambda, \mu_1, \mu_2} g(\lambda, \mu_1, \mu_2) = max (-(4\lambda + 3\mu_1 + 2\mu_2)) = -min(4\lambda + 3\mu_1 + 2\mu_2) = \frac{-73}{9} = d^*$

**Ответ:**
$p^* = -2$\\
$d^* = \frac{-73}{9}$
$\epsilon = p^* - d^* = \frac{35}{9}$
\end{document}